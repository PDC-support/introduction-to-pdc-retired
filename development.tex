\section{Development}
\subsection{Building}
\begin{frame}[fragile]
\frametitle{Compiling, Linking and Running Applications}
\framesubtitle{on HPC clusters}
 \begin{description}
    \item [source code] C / C++ / Fortran ( \verb|.c, .cpp, .f90, .h|  )
    \item [compile] Cray/GNU/AMD compilers
    \item [assemble] into machine code (object files: \verb|.o, .obj| )
    \item [link] Static Libraries (\verb|.lib, .a|  ) \\ Shared Library (\verb|.dll, .so| ) \\ Executables (\verb|.exe, .x| )
    \item ~ 
    \item [request allocation] submit job request to SLURM queuing system \\ \verb|salloc/sbatch|
    \item [run] application on scheduled resources \\ \verb|srun|
 \end{description}
\end{frame}

\subsection{Modules}
\begin{frame}[fragile]
\frametitle{Modules}
\framesubtitle{Using Lmod}

\begin{exampleblock}{List loaded modules}
  \begin{verbatim}
  ml
  \end{verbatim}
\end{exampleblock}

\begin{exampleblock}{List available modules}
  \begin{verbatim}
  ml avail
  \end{verbatim}
\end{exampleblock}

\begin{exampleblock}{Load modules}
  \begin{verbatim}
  ml <software_name>
  \end{verbatim}
\end{exampleblock}

\begin{exampleblock}{Unload modules}
  \begin{verbatim}
  ml -<software_name>
  \end{verbatim}
\end{exampleblock}
\end{frame}


\begin{frame}[fragile]
\frametitle{Modules }
\framesubtitle{Displaying modules}
\begin{exampleblock}{\$ ml}
\scriptsize
\begin{verbatim}
Currently Loaded Modulefiles:
  1) craype-x86-rome
  ...
  10) cray-libsci/21.08.1.2
\end{verbatim}
\end{exampleblock}

\begin{exampleblock}{\$ ml avail [software\_name]}
\scriptsize
\begin{verbatim}
---------------- /opt/cray/pe/lmod/modulefiles/cpu/x86-rome/1.0 ---------------
     cray-fftw/3.3.8.10    cray-fftw/3.3.8.11    cray-fftw/3.3.8.12 (D)
\end{verbatim}
\end{exampleblock}

\begin{exampleblock}{\$ module show [software\_name]}
\scriptsize
\begin{verbatim}
...
whatis("FFTW 3.3.8.12 - Fastest Fourier Transform in the West")
setenv("FFTW_VERSION","3.3.8.12")
setenv("CRAY_FFTW_VERSION","3.3.8.12")
setenv("FFTW_ROOT","/opt/cray/pe/fftw/3.3.8.12/x86_rome")
...
\end{verbatim}
\end{exampleblock}
\end{frame}

\subsection{Programming environments}
\begin{frame}[fragile]
  \frametitle{Programming Environment Modules}

\begin{columns}[t]
\column{.7\textwidth}
  \begin{description}
  \item [Cray] \verb|$ ml PrgEnv-cray|
  \item [GNU] \verb|$ ml PrgEnv-gnu|
  \item [AMD] \verb|$ ml PrgEnv-aocc|
  \end{description}
   % \item Module cray-libsci provides BLAS, LAPACK, BLACS, and SCALAPACK
   % \item Module cray-mpich provides MPI
\column{.3\textwidth}
    \begin{verbatim}
$ cc	source.c
$ CC	source.cpp
$ ftn	source.F90
  \end{verbatim}
\end{columns}
  \begin{exampleblock}{Compiler wrappers : \alert{\textbf{cc} \textbf{CC} \textbf{ftn}}}
    \alert{Advantages}\\
    Compiler wrappers will automatically 
    \begin{itemize}
      \item link to BLAS, LAPACK, BLACS, SCALAPACK, FFTW\\
      \item use MPI wrappers\\
    \end{itemize}
    \alert{Disadvantage}\\
    Sometimes you need to edit Makefiles which are not designed for Cray 
\end{exampleblock}
\end{frame}


\begin{frame}[fragile]
\frametitle{Programming Environment Modules}
    \begin{exampleblock}{The PDC module}
        .. write me ..
    \end{exampleblock}
\end{frame}
