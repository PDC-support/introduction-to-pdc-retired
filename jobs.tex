\section{Running jobs}
\frame{
\frametitle{How to run programs}
\begin{itemize}
  \item After login we are on a \textit{login node} used only for:
  \begin{itemize}
    \item submitting jobs,
    \item editing files,
    \item compiling small programs,
    \item other computationally light tasks.
  \end{itemize}
  \item \alert{Never run calculations interactively on the login node} 
  \item Instead, request compute resources \textit{interactively} or via \textit{batch script} \\~

  \item All jobs must be connected to a time allocation
  \item For courses, PDC sets up a \textit{reservation} for resources \\~

  \item To manage the workload on the clusters, PDC uses a queueing/batch system

  \end{itemize}
 }


\subsection{SLURM}
\frame{
\frametitle{SLURM workload manager}
\framesubtitle{Simple Linux Utility for Resource Management}

\begin{itemize}
 \item Open source, fault-tolerant, and highly scalable cluster management and job scheduling system
 \begin{itemize}
  \item \alert{Allocates} exclusive and/or non-exclusive access to \alert{resources} for some duration of time
  \item Provides a framework for \alert{starting}, \alert{executing}, and \alert{monitoring} work on the set of allocated nodes
  \item \alert{Arbitrates contention} for resources by managing a queue 
 \end{itemize}
 \item Job Priority computed based on 
 \begin{description}
    \item [Age] the length of time a job has been waiting
    \item [Fair-share] the difference between the portion of the computing resource that has been promised and the amount of resources that has been consumed
    \item [Job size] the number of nodes or CPUs a job is allocated
    \item [Partition] a factor associated with each node partition
%    \item [QOS] a factor associated with each Quality Of Service
 \end{description}
\end{itemize}
}

\subsection*{SLURM commands}

\begin{frame}[fragile]
\frametitle{Interactive session \hfill \alert{\textbf{salloc}}}

\begin{exampleblock}{Request an interactive allocation of resources}
  \begin{verbatim}
  $ salloc -A <account> -t <d-hh:mm:ss> -N <nodes>
  salloc: Granted job allocation 123456
  \end{verbatim}
\end{exampleblock}

\begin{exampleblock}{Run application on compute nodes}
  \begin{verbatim}
  $ srun -n <PEs> ./binary.x
  #PEs: number of processing elements (MPI processes)
  \end{verbatim}
\end{exampleblock}
  
\end{frame}

\begin{frame}[fragile]
\frametitle{Launch batch jobs \hfill  \alert{\textbf{sbatch}}}
\begin{exampleblock}{Submit the job to SLURM queue}
  \begin{verbatim}
$ sbatch <script>
Submitted batch job 123456
  \end{verbatim}
\end{exampleblock}

\scriptsize
The script should contain all necessary data to identify the account and requested resources 
\begin{exampleblock}{Example of request to run myexe for 1 hour on 2 nodes}
\begin{verbatim}
#!/bin/bash

#SBATCH -A summer-2019
#SBATCH -J myjob
#SBATCH -t 01:00:00
#SBATCH --nodes=2
#SBATCH --ntasks-per-node=128

srun ./myexe > my_output_file
\end{verbatim}
\end{exampleblock}

\end{frame}

\begin{frame}[fragile]
\frametitle{Monitoring and/or cancelling running jobs }
\begin{alertblock}{\textbf{squeue} -u  \$USER}
  Displays all queue and/or running jobs that belong to the user
\tiny
  \begin{verbatim}
user@dardel$ squeue -u user
 JOBID     USER ACCOUNT           NAME  ST REASON    START_TIME                TIME  TIME_LEFT NODES
 63519   user 20XX-X-XX      test-run1   R None      2021-11-15T08:15:24    6:09:42   17:49:18     2
 63757   user 20XX-X-XX      test-run2   R None      2021-11-15T11:14:20    3:10:46   20:48:14     8
  \end{verbatim}
\end{alertblock}

\begin{alertblock}{\textbf{scancel} [job]}
Stops a running job or removes a pending one from the queue
\tiny
  \begin{verbatim}
user@dardel$ scancel 63519
salloc: Job allocation 63519 has been revoked.

user@dardel$ squeue -u user
 JOBID     USER ACCOUNT           NAME  ST REASON    START_TIME                TIME  TIME_LEFT NODES
 63757   user 20XX-X-XX      test-run2   R None      2021-11-15T11:14:20    3:10:46   20:48:14     8
  \end{verbatim}
\end{alertblock}
\end{frame}
